%-------------------------
% Resume in Latex
% Author : Gennadii Chursov
% Based off of: https://github.com/sb2nov/resume and https://www.overleaf.com/latex/templates/jakes-resume/syzfjbzwjncs
% License : MIT
%------------------------

% Curriculum Unificado: Mismo documento para obtener publicaciones, lista de platicas
\documentclass[letterpaper,10pt]{article}

%\usepackage[
%colorlinks,
%linkcolor=blue,
%citecolor=green,
%pdftex=true,
%bookmarks=true,
%bookmarksopen=false,
%pdftitle={Listado de Publicaciones},
%pdfsubject={Reporte Final de Proyecto},
%pdfkeywords={FPGAs, Reconfigurable Computing, Machine Learning, Computer Vision},
%pdfauthor={Marco Aurelio Nu\~no Maganda}
%]{hyperref}

\usepackage{pdfpages}

\usepackage{latexsym}
\usepackage[empty]{fullpage}
\usepackage{titlesec}
\usepackage{marvosym}
%\usepackage[usenames,dvipsnames]{color}
\usepackage{verbatim}
\usepackage{enumitem}
\usepackage[colorlinks,
linkcolor=blue,
citecolor=green]{hyperref}
\usepackage{fancyhdr}
\usepackage[spanish]{babel}
\usepackage{tabularx}
\input{glyphtounicode}


\pagestyle{fancy}
\fancyhf{} % clear all header and footer fields
\fancyfoot{}
\renewcommand{\headrulewidth}{0pt}
\renewcommand{\footrulewidth}{0pt}

% Adjust margins
\addtolength{\oddsidemargin}{-0.5in}
\addtolength{\evensidemargin}{-0.5in}
\addtolength{\textwidth}{1in}
\addtolength{\topmargin}{-.5in}
\addtolength{\textheight}{1.0in}

\urlstyle{same}

\raggedbottom
\raggedright
\setlength{\tabcolsep}{0in}

% Sections formatting
\titleformat{\section}{
  \vspace{-4pt}\scshape\raggedright\large
}{}{0em}{}[\color{black}\titlerule \vspace{-5pt}]

% Ensure that generate pdf is machine readable/ATS parsable
\pdfgentounicode=1

%-------------------------
% Custom commands
\newcommand{\resumeItem}[1]{
  \item\small{
    {#1 \vspace{-2pt}}
  }
}

\newcommand{\resumeSubheading}[4]{
  \vspace{-2pt}\item
    \begin{tabular*}{0.97\textwidth}[t]{l@{\extracolsep{\fill}}r}
      \textbf{#1} & #2 \\
      \textit{\small#3} & \textit{\small #4} \\
    \end{tabular*}\vspace{-7pt}
}

\newcommand{\resumeSubSubheading}[2]{
    \item
    \begin{tabular*}{0.97\textwidth}{l@{\extracolsep{\fill}}r}
      \textit{\small#1} & \textit{\small #2} \\
    \end{tabular*}\vspace{-7pt}
}

\newcommand{\resumeProjectHeading}[2]{
    \item
    \begin{tabular*}{0.97\textwidth}{l@{\extracolsep{\fill}}r}
      \small#1 & #2 \\
    \end{tabular*}\vspace{-7pt}
}

\newcommand{\resumeSubItem}[1]{\resumeItem{#1}\vspace{-4pt}}

\renewcommand\labelitemii{$\vcenter{\hbox{\tiny$\bullet$}}$}

\newcommand{\resumeSubHeadingListStart}{\begin{itemize}[leftmargin=0.15in, label={}]}
\newcommand{\resumeSubHeadingListEnd}{\end{itemize}}
\newcommand{\resumeItemListStart}{\begin{itemize}}
\newcommand{\resumeItemListEnd}{\end{itemize}\vspace{-5pt}}




\usepackage[backend=biber,sorting=ddatent,style=ieee, defernumbers=true, natbib=true, isbn=true, maxnames=99, maxbibnames=99]{biblatex}

\DeclareSortingScheme{ddatent}{
    \sort{
        \field{presort}
    }
    \sort[final]{
        \field{sortkey}
    }
    \sort[direction=descending]{
        \field[strside=left,strwidth=4]{sortyear}
        \field[strside=left,strwidth=4]{year}
        \literal{9999}
    }
    \sort[direction=descending]{
        \field[padside=left,padwidth=2,padchar=0]{month}
        \literal{00}
    }
    \sort[direction=descending]{
        \field[padside=left,padwidth=2,padchar=0]{day}
        \literal{00}
    }
    \sort[direction=descending]{
        \field[padside=left,padwidth=4,padchar=0]{volume}
        \literal{9999}
    }
    \sort{
        \name{sortname}
        \name{author}
        \name{editor}
        \name{translator}
        \field{sorttitle}
        \field{title}
    }
    \sort{
        \field{sorttitle}
        \field{title}
    }
}

% NO dejar lineas en blanco porque la se;orita se enoja!!
\DeclareSourcemap{
  \maps[datatype=bibtex, overwrite]{
    \map{
      \perdatasource{TesisDirigidasMaestria.bib}
      \step[fieldset=keywords, fieldvalue={, }, appendstrict]
      \step[fieldset=keywords, fieldvalue=one, append]
    }
    \map{
      \perdatasource{TesisDirigidasLicenciatura.bib}
      \step[fieldset=keywords, fieldvalue={, }, appendstrict]
      \step[fieldset=keywords, fieldvalue=two, append]
    }
    \map{
      \perdatasource{Revistas.bib}
      \step[fieldset=keywords, fieldvalue={, }, appendstrict]
      \step[fieldset=keywords, fieldvalue=three, append]
    }
    \map{
      \perdatasource{CongresosArb.bib}
      \step[fieldset=keywords, fieldvalue={, }, appendstrict]
      \step[fieldset=keywords, fieldvalue=four, append]
    }
    \map{
      \perdatasource{CongresosEspanol.bib}
      \step[fieldset=keywords, fieldvalue={, }, appendstrict]
      \step[fieldset=keywords, fieldvalue=five, append]
    }
    \map{
      \perdatasource{DivulgacionCienciaTecnologia.bib}
      \step[fieldset=keywords, fieldvalue={, }, appendstrict]
      \step[fieldset=keywords, fieldvalue=DCyT, append]
    }
    \map{
      \perdatasource{DifusionEnEventos.bib}
      \step[fieldset=keywords, fieldvalue={, }, appendstrict]
      \step[fieldset=keywords, fieldvalue=DEPA, append]
    }
    \map{
      \perdatasource{Cursos.bib}
      \step[fieldset=keywords, fieldvalue={, }, appendstrict]
      \step[fieldset=keywords, fieldvalue=CURSOS, append]
    }
    \map{
      \perdatasource{ColaboradorProyectos.bib}
      \step[fieldset=keywords, fieldvalue={, }, appendstrict]
      \step[fieldset=keywords, fieldvalue=COLABORADOR, append]
    }
  }
}

% Count total number of entries in each refsection
\AtDataInput{%
  \csnumgdef{entrycount:\therefsection}{%
    \csuse{entrycount:\therefsection}+1}}

% Print the labelnumber as the total number of entries in the
% current refsection, minus the actual labelnumber, plus one
\DeclareFieldFormat{labelnumber}{\mkbibdesc{#1}}    
\newrobustcmd*{\mkbibdesc}[1]{%
  \number\numexpr\csuse{entrycount:\therefsection}+1-#1\relax}

\addbibresource{TesisDirigidasMaestria.bib}
\addbibresource{TesisDirigidasLicenciatura.bib}
\addbibresource{Revistas.bib}
\addbibresource{CongresosArb.bib}
\addbibresource{CongresosEspanol.bib}
\addbibresource{DivulgacionCienciaTecnologia.bib}
\addbibresource{DifusionEnEventos.bib}
\addbibresource{Cursos.bib}
\addbibresource{ColaboradorProyectos.bib}


\newcounter{num}
\setcounter{num}{0} % 1=> LINKS DE EVIDENCIA, 0=> NO LINKS DE EVIDENCIA

%-------------------------------------------
%%%%%%  RESUME STARTS HERE  %%%%%%%%%%%%%%%%%%%%%%%%%%%%


\begin{document}

\begin{center}
    \textbf{\Large \scshape Marco Aurelio Nuño Maganda} \\ \vspace{1pt}
    \textit{Profesor Investigador en Ciencias de la Computación} \\
    Última Actualización: \today \\
\end{center}

\begin{minipage}{0.98 \textwidth} % 0.5 \textwidth = 50% ancho de página
Fecha de Nacimiento: \textbf{12 de Agosto de 1978} \\
Lugar de Nacimiento: \textbf{Mexico, D. F.}\\
Nacionalidad: \textbf{Mexicana}\\
Telefono: 834-147-28-35  \\
ORCID: \ \ \textbf{\url{https://orcid.org/0000-0003-0102-8227}} \\
%Mendeley: \\
%\textbf{\url{https://www.mendeley.com/profiles/marco-aurelio-nuo-maganda/}} \\
Scopus: \ \ \textbf{\url{https://www.scopus.com/authid/detail.uri?authorId=15846253000}} \\
Sci Profile:\ \ \textbf{\url{https://sciprofiles.com/profile/449039}} \\
Google Scholar:\ \ \textbf{\url{https://scholar.google.com/citations?user=hSfqx4EAAAAJ&hl=en}}\\
Research Gate:\ \ \textbf{\url{https://www.researchgate.net/profile/Marco_Nuno}}\\
Publons:\ \  \textbf{\url{https://publons.com/researcher/1603078/marco-aurelio-nuno-maganda}}\\
\end{minipage}


\section{Educación}
 \begin{itemize}[leftmargin=0.15in, label={}]
\item 2006 -- 2009: \textbf{Doctorado en Ciencias con Especialidad en Ciencias Computacionales}. 
    Instituto Nacional de Astrofísica, Óptica y Electrónica (INAOE), Puebla, México.
\ifthenelse{\(\value{num}=1\)}{\hyperlink{./../Evidencias/sensors-18-02202-v2.pdf.1}{Titulo y Cédula}}{}
%
\item 2001 -- 2003: \textbf{Maestría en Ciencias con Especialidad en Ciencias Computacionales}
    Instituto Nacional de Astrofísica, Óptica y Electrónica (INAOE), Puebla, México. 
\ifthenelse{\(\value{num}=1\)}{\hyperlink{./../Evidencias/sensors-18-02202-v2.pdf.1}{Titulo y Cédula}}{}

\item 1996 -- 2001: \textbf{Licenciatura en Ingeniería en Computación}.
	Universidad Americana de Acapulco (Incorporada a la UNAM), Acapulco, México, Graduado con Mención Honorífica. 
\ifthenelse{\(\value{num}=1\)}{\hyperlink{./../Evidencias/sensors-18-02202-v2.pdf.1}{Titulo y Cédula}}{}
 \end{itemize}


\section{Pertenencia al Sistema Nacional de Investigadores (SNI)}
 \begin{itemize}[leftmargin=0.15in, label={}]
\item 2023 -- 2027: Miembro del \textbf{Sistema Nacional de Investigadores (SNI)}, Nivel I (Enero 2023 a Diciembre de 2027). \ifthenelse{\(\value{num}=1\)}{\hyperlink{./../Evidencias/sensors-18-02202-v2.pdf.1}{Titulo y Cédula}}{}
\item 2020 -- 2023: Miembro del \textbf{Sistema Nacional de Investigadores (SNI)}, Nivel I (Enero 2020 a Diciembre de 2022). \ifthenelse{\(\value{num}=1\)}{\hyperlink{./../Evidencias/sensors-18-02202-v2.pdf.1}{Titulo y Cédula}}{}
\item 2014 -- 2016: Miembro del \textbf{Sistema Nacional de Investigadores (SNI)}, Nivel Candidato (Enero 2014 a Diciembre de 2016). \ifthenelse{\(\value{num}=1\)}{\hyperlink{./../Evidencias/sensors-18-02202-v2.pdf.1}{Titulo y Cédula}}{}
 \end{itemize}



\section{Experiencia Profesional y Docente}
 \begin{itemize}[leftmargin=0.15in, label={}]
\item 2009 -- 2024: \textbf{Profesor de Tiempo Completo}, Universidad Politécnica de Victoria (UPV), Ciudad Victoria, Tamaulipas.  Principales funciones: Catedrático de Asignatura en Nivel Maestría y Licenciatura (Carrera: Ingenier\'ia en Tecnologías de la Información).
\ifthenelse{\(\value{num}=1\)}{\hyperlink{./../Evidencias/sensors-18-02202-v2.pdf.1}{Titulo y Cédula}}{}

\item 2005 -- 2009: \textbf{Profesor de Asignatura}, Instituto Tecnol\'ogico Superior de Atlixco (ITSA), Atlixco, Puebla. Principales funciones: Catedrático de Asignatura en Nivel Licenciatura (Carrera: Ingenier\'ia en Sistemas Computacionales). \ifthenelse{\(\value{num}=1\)}{\hyperlink{./../Evidencias/sensors-18-02202-v2.pdf.1}{Titulo y Cédula}}{}

\item 2006 -- 2006: \textbf{Profesor de Asignatura}, Universidad Polit\'ecnica de Puebla (UPP), Cholula, Puebla. Principales funciones: Catedrático de Asignatura en Nivel Licenciatura (Carrera: Lic en Inform\'atica) 
\ifthenelse{\(\value{num}=1\)}{\hyperlink{./../Evidencias/sensors-18-02202-v2.pdf.1}{Titulo y Cédula}}{}

\item 2004 -- 2005: \textbf{Programador de Base de Datos}, Centro Regional para la Competitividad Empresarial (CRECE)-OAXACA, Oaxaca, Oaxaca. Principales funciones: An\'alisis y Dise\~no de aplicaciones de Bases de Datos. Lenguaje de Programaci\'on utilizado: Visual Fox Pro. 
\ifthenelse{\(\value{num}=1\)}{\hyperlink{./../Evidencias/sensors-18-02202-v2.pdf.1}{Titulo y Cédula}}{}

\item 2003 -- 2005: \textbf{Programador de Aplicaciones FPGA}, Instituto Nacional de Astrofísica, Óptica y Electrónica (INAOE), Cholula, Puebla. Principales funciones: Análisis y Diseño de Aplicaciones Implementadas en FPGAs para resolver problemas de Vision por Computadora. Programación en Lenguajes de Descripcion de Hardware y uso de Herramientas CAD para implementación de Aplicaciones en FPGA. 
\ifthenelse{\(\value{num}=1\)}{\hyperlink{./../Evidencias/sensors-18-02202-v2.pdf.1}{Titulo y Cédula}}{}

 \end{itemize}

% De uso exclusivo UPV
\section{Cursos}
\begin{refsection}[Cursos]
\nocite{*}
\printbibliography[heading={none}, keyword=CURSOS, resetnumbers=true]
\end{refsection}

% De uso exclusivo UPV
\section{Colaborador en Proyectos de Investigacion y Desarrollo Tecnológico}
\begin{refsection}[ColaboradorPI_DT]
\nocite{*}
\printbibliography[heading={none}, keyword=COLABORADOR, resetnumbers=true]
\end{refsection}



% De uso exclusivo UPV - Duplicados entre Congresos, Cap de Libro y Revistas
\section{Difusión en Congresos Especializados}
\begin{refsection}[DifusionEnEventos]
\nocite{*}
\printbibliography[heading={none}, keyword=DEPA, resetnumbers=true]
\end{refsection}



\section{Divulgación de la Ciencia y la Tecnología}
\begin{refsection}[DivulgacionCienciaTecnologia]
\nocite{*}
%\printbibliography[title={Divulgacion de Ciencia y Tecnologia}, keyword=DCyT, resetnumbers=true]
\printbibliography[heading={none}, keyword=DCyT, resetnumbers=true]
\end{refsection}


\section{Articulos Publicados en Revistas Arbitradas e Indexadas}
\begin{refsection}[Revistas]
\nocite{*}
%\printbibliography[title={First Bibliography}, keyword=three, resetnumbers=true]
\printbibliography[heading={none}, keyword=three, resetnumbers=true]
\end{refsection}





\section{Articulos Publicados en Memorias de Congresos en Inglés}
\begin{refsection}[CongresosArb]
\nocite{*}
%\printbibliography[title={Ponencias en Congresos Internacionales en Inglés}, keyword=four, resetnumbers=true]
\printbibliography[heading={none}, keyword=four, resetnumbers=true]
\end{refsection}



\section{Articulos Publicados en Memorias de Congresos en Español}
\begin{refsection}[CongresosEspanol]
\nocite{*}
%\printbibliography[title={Ponencias en Congresos Internacionales en Espanol}, keyword=five, resetnumbers=true]
\printbibliography[heading={none}, keyword=five, resetnumbers=true]
\end{refsection}



\section{Tesis de Maestría Dirigidas}
\begin{refsection}[TesisDirigidasMaestria]
\nocite{*}
%\printbibliography[title={First Bibliography}, keyword=one, resetnumbers=true]
\printbibliography[heading={none}, keyword=one, resetnumbers=true]
\end{refsection}

\section{Tesis de Licenciatura Dirigidas}
\begin{refsection}[TesisDirigidasLicenciatura]
\nocite{*}
%\printbibliography[title={First Bibliography}, keyword=two, resetnumbers=true]
\printbibliography[heading={none}, keyword=two, resetnumbers=true]
\end{refsection}



% De requerir evidencias, deberan incluirse con sus respectivos
\ifthenelse{\(\value{num}=1\)}{\clearpage}{}
%\includepdf[pages=1-,link=true]{./../Evidencias/applsci-10-06846.pdf}
\ifthenelse{\(\value{num}=1\)}{\includepdf[pages={1},link=true]{./../Evidencias/applsci-10-06846.pdf}}{}
%\includepdf[pages=1-,link=true]{./../Evidencias/sensors-18-02202-v2.pdf}
\ifthenelse{\(\value{num}=1\)}{\includepdf[pages={1},link=true]{./../Evidencias/sensors-18-02202-v2.pdf}}{}


\end{document}


\end{document}




